\section{Test - Mengentheoretische Grundlagen}

\subsection*{explizite Mengendarstellungen}

%BEGIN FRAGE
\frage{Sei
$M = \{ n \in \mathbb{N} \ | \ n \leq 23  \ \text{und wenn} \ n \ \text{durch} \ 2 \ \text{teilbar ist, so gilt} \ n > 16 \}$.
Dann ist die explizite Darstellung von $M$:
}

\antwort{
\typ{Multiple Choice, Kprim}
\level{leicht, mittel, schwer}
\begin{description}
\antwfalse $M = \{0,1,2,3,4,5,6,7,8,9,10,11,12,13,14,15,16,17,18,19,20,21,22,23\}$.
\antwfalse $M = \{1,3,5,7,9,11,13,15,16,17,18,19,20,21,22,23\}$.
\antwright $M = \{1,3,5,7,9,11,13,15,17,18,19,20,21,22,23\}$.
\antwfalse $M = \{0,1,3,5,7,9,11,13,15,17,18,19,20,21,22,23\}$.
\end{description}

\feedbackright{TODO}

\feedbackfalse{TODO}
}
%END FRAGE


%BEGIN FRAGE
\frage{Sei
$M = \{ n \in \mathbb{N} \ | \ n \leq 13 \ \text{und es gibt} \ m \in \mathbb{N}  \ \text{so, dass} \ n = 2m \}$.
Dann ist die explizite Darstellung von $M$:
}

\antwort{
\typ{Multiple Choice, Kprim}
\level{leicht, mittel, schwer}
\begin{description}
\antwright $M = \{0,2,4,6,8,10,12\}$.
\antwfalse $M = \{1,3,5,7,9,11,13\}$.
\antwfalse $M = \{0,1,2,3,4,5,6,7,8,9,10,11,12,13\}$.
\antwfalse $M = \{2,4,6,8,10,12\}$.
\end{description}

\feedbackright{TODO}

\feedbackfalse{TODO}
}
%END FRAGE

%BEGIN FRAGE
\frage{Sei
$M = \{ n \in \mathbb{N} \ | \ n \leq 12 \ \text{und es gibt} \ m \in \mathbb{N}  \
\text{mit}\  m > 2 \ \text{so, dass} \ n = 2m \}$.
Dann ist die explizite Darstellung von $M$:
}

\antwort{
\typ{Multiple Choice, Kprim}
\level{leicht, mittel, schwer}
\begin{description}
\antwfalse $M = \{1,3,5,7,9,11,13\}$.
\antwfalse $M = \{0,6,8,10,12\}$.
\antwright $M = \{6,8,10,12\}$.
\antwfalse $M = \{4,6,8,10,12\}$.
\end{description}

\feedbackright{TODO}

\feedbackfalse{TODO}
}
%END FRAGE




%BEGIN FRAGE
\frage{Sei
$M = \{ n \in \mathbb{N} \ | \ n \leq 13 \ \text{und es gibt} \ m \in \mathbb{N}  \ \text{so, dass} \ n = 2m \ \text{und es gibt} \ k \in \mathbb{N}  \ \text{so, dass} \ n = 3k \}$.
Dann ist die explizite Darstellung von $M$:
}

\antwort{
\typ{Multiple Choice, Kprim}
\level{leicht, mittel, schwer}
\begin{description}
\antwfalse $M = \{1,2,3,4,5,7,8,9,10,11,13\}$.
\antwfalse $M = \{6,12\}$.
\antwright $M = \{0,6,12\}$.
\antwfalse $M = \{0,2,3,4,6,8,9,10,12\}$.
\end{description}

\feedbackright{TODO}

\feedbackfalse{TODO}
}
%END FRAGE



%BEGIN FRAGE
\frage{Sei
$M = \{ n \in \mathbb{N} \ | \ n \leq 13 \ \text{und es gibt} \ m \in \mathbb{N}  \ \text{so, dass} \ n = 2m \ \text{und es gibt kein} \ k \in \mathbb{N}  \ \text{so, dass} \ n = 3k \}$.
Dann ist die explizite Darstellung von $M$:
}

\antwort{
\typ{Multiple Choice, Kprim}
\level{leicht, mittel, schwer}
\begin{description}
\antwfalse $M = \{0,1,2,3,4,5,6,7,8,9,10,11,12,13\}$.
\antwfalse $M = \{0,2,4,6,8,10,12\}$.
\antwfalse $M = \{0,2,4,8,10\}$.
\antwright $M = \{2,4,8,10\}$.
\end{description}

\feedbackright{TODO}

\feedbackfalse{TODO}
}
%END FRAGE


%BEGIN FRAGE
\frage{Sei
$M = \{ n \in \mathbb{N} \ | \ n \leq 13 \ \text{und es gibt} \ m \in \mathbb{N}  \ \text{so, dass} \ n = 2m \ \text{und } \ n = 3m \}$.
Dann ist die explizite Darstellung von $M$:
}

\antwort{
\typ{Multiple Choice, Kprim}
\level{leicht, mittel, schwer}
\begin{description}
\antwright $M = \emptyset$.
\antwfalse $M = \{0\}$.
\antwfalse $M = \{0,2,3,4,6,8,9,10,12\}$.
\antwfalse $M = \{0,1,2,3,4,5,6,7,8,9,10,11,12,13\}$.
\end{description}

\feedbackright{TODO}

\feedbackfalse{TODO}
}
%END FRAGE

\subsection*{deskriptive Mengendarstellungen}

%BEGIN FRAGE
\frage{Sei $M = \{0,1,2,4 \}$. Wählen Sie alle deskriptiven Darstellungen von $M$.}

\antwort{
\typ{Multiple Choice, Kprim}
\level{leicht, mittel, schwer}
\begin{description}
\antwright $M = \{ n \in \mathbb{N} \ | \ n < 3 \ \text{oder} \ 2 \cdot n = 8\}$.
\antwfalse $M = \{ n \in \mathbb{N} \ | \ n \leq 3 \ \text{oder} \ n = 4\}$.
\antwfalse $M = \{ n \in \mathbb{N} \ | \ \text{es gibt} \ m \in \mathbb{N} \ \text{mit} \ m \geq 3 \ \text{und} \ m \neq 4 \ \text{und} \  n \cdot m = 12\}$.
\antwfalse $M = \{ n \in \mathbb{N} \ | \ \text{es gibt} \ m \in \{ 0, 3, 6, 9, 12 \} \ \text{mit} \ 3 \cdot n = m\}$.
\end{description}

\feedbackright{TODO}

\feedbackfalse{TODO}
}
%END FRAGE

%BEGIN FRAGE
\frage{Sei $M = \{1,3,5,7,9 \}$. Wählen Sie alle deskriptiven Darstellungen von $M$.}

\antwort{
\typ{Multiple Choice, Kprim}
\level{leicht, mittel, schwer}
\begin{description}
\antwfalse $M = \{ n \in \mathbb{N} \ | \ \text{es gibt} \ m \in \mathbb{N} \ \text{mit} \ 0 < m < 5 \ \text{und} \ n = 2 \cdot m + 1 \}$.
\antwright $M = \{ n \in \mathbb{N} \ | \ \text{es gibt} \ m \in \mathbb{N} \ \text{mit} \ 0 \leq m < 5 \ \text{und} \ n = 2 \cdot m + 1 \}$.
\antwright $M = \{ n \in \mathbb{N} \ | \ \text{es gibt} \ m \in \mathbb{N} \ \text{mit} \ 0 < m < 6 \ \text{und} \ n = 2 \cdot m - 1 \}$.
\antwfalse $M = \{ n \in \mathbb{N} \ | \ \text{es gibt} \ m \in \mathbb{N} \ \text{mit} \ 0 \leq m < 4 \ \text{und} \ n = 2 \cdot m + 3 \}$.
\end{description}

\feedbackright{TODO}

\feedbackfalse{TODO}
}
%END FRAGE

%BEGIN FRAGE
\frage{Sei $M = \{1,3,5,6,8,10 \}$. Wählen Sie alle deskriptiven Darstellungen von $M$.}

\antwort{
\typ{Multiple Choice, Kprim}
\level{leicht, mittel, schwer}
\begin{description}
\antwright $M = \{ n \in \mathbb{N} \ | \ n \leq 10 \ \text{und es gilt} \ n \geq 6 \ \text{genau dann, wenn es kein} \ m \in \mathbb{N} \ \text{mit} \ n = 2 \cdot m + 1 \ \text{gibt} \}$.
\antwright $M = \{ n \in \mathbb{N} \ | \ n < 11 \ \text{und es gilt} \ n < 6 \ \text{genau dann, wenn es ein} \ m \in \mathbb{N} \ \text{mit} \ n = 2 \cdot m + 1 \ \text{gibt} \}$.
\antwright $M = \{ n \in \mathbb{N} \ | \ n \leq 10 \ \text{und es gilt} \ n \geq 6 \ \text{genau dann, wenn es ein} \ m \in \mathbb{N} \ \text{mit} \ n = 2 \cdot m \ \text{gibt} \}$.
\antwfalse $M = \{ n \in \mathbb{N} \ | \ n < 11 \ \text{und es gilt} \ n \leq 6 \ \text{genau dann, wenn es kein} \ m \in \mathbb{N} \ \text{mit} \ n = 2 \cdot m \ \text{gibt} \}$.
\end{description}

\feedbackright{TODO}

\feedbackfalse{TODO}
}
%END FRAGE

\subsection*{Konstruktion aus Mengen}

%BEGIN FRAGE
\frage{Seinen $M = \{ 1,2,3,5\}$ und $N = \{0,1,4,5\}$. Dann gelten:}

\antwort{
\typ{Multiple Choice, Kprim}
\level{leicht, mittel, schwer}
\begin{description}
\antwfalse $M \subseteq N$ oder $N \subseteq M$.
\antwfalse $M \cup N \subseteq M$ oder $M \cup N \subseteq N$.
\antwright $M \cap N \subseteq M$ oder $M \cap N \subseteq N$.
\antwright $M \backslash N \subseteq M \cap N$ oder $M \backslash N \subseteq M \cup N$.
\end{description}

\feedbackright{TODO}

\feedbackfalse{TODO}
}
%END FRAGE

%BEGIN FRAGE
\frage{Seinen $M = \{ 1,2,3,5\}$ und $N = \{n \in \mathbb{N} \ | \ n \leq 8 \ \text{und es gibt} \ m \in \mathbb{N} \ \text{mit} \ 2 \cdot m = n\}$. Dann gelten:}

\antwort{
\typ{Multiple Choice, Kprim}
\level{leicht, mittel, schwer}
\begin{description}
\antwfalse $M \cup N = \{0,2,4,6,8 \}$.
\antwright $M \cup N = \{0,1,2,3,4,5,6,8 \}$.
\antwfalse $M \cup N = \{1,2,3,4,5,6,8 \}$.
\antwfalse $M \cup N = \{1,2,3,4,5,6,7,8 \}$.
\end{description}

\feedbackright{TODO}

\feedbackfalse{TODO}
}
%END FRAGE


%BEGIN FRAGE
\frage{Seinen $M = \{ 1,2,3,5\}$ und $N = \{n \in \mathbb{N} \ | \ n \leq 8 \ \text{und es gibt} \ m \in \mathbb{N} \ \text{mit} \ 2 \cdot m = n\}$. Dann gelten:}

\antwort{
\typ{Multiple Choice, Kprim}
\level{leicht, mittel, schwer}
\begin{description}
\antwfalse $M \cap N = \{0,2 \}$.
\antwfalse $M \cap N = \{1,3,5 \}$.
\antwright $M \cap N = \{2\}$.
\antwfalse $M \cap N = \{0,1,2,3,4,5,6,7,8 \}$.
\end{description}

\feedbackright{TODO}

\feedbackfalse{TODO}
}
%END FRAGE

%BEGIN FRAGE
\frage{Seinen $M = \{ 1,2,3,5\}$ und $N = \{n \in \mathbb{N} \ | \ n \leq 8 \ \text{und es gibt} \ m \in \mathbb{N} \ \text{mit} \ 2 \cdot m = n\}$. Dann gelten:}

\antwort{
\typ{Multiple Choice, Kprim}
\level{leicht, mittel, schwer}
\begin{description}
\antwfalse $M \backslash N = \{1,5 \}$.
\antwfalse $M \backslash N = \{0,1,3,5 \}$.
\antwright $N \backslash M = \{0,4,6,8\}$.
\antwfalse $N \backslash M = \{0,4,6,7,8 \}$.
\end{description}

\feedbackright{TODO}

\feedbackfalse{TODO}
}
%END FRAGE


%BEGIN FRAGE
\frage{Sei $\mathcal{M} = \{ \emptyset, \{0,1,2\}, \{1,2\}, \{1,2,3\}\}$. Dann gelten:}

\antwort{
\typ{Multiple Choice, Kprim}
\level{leicht, mittel, schwer}
\begin{description}
\antwfalse $\bigcup \mathcal{M} = \emptyset$.
\antwfalse $\bigcup \mathcal{M} = \{ 0\}$.
\antwfalse $\bigcup \mathcal{M} = \{ 1,2\}$.
\antwright $\bigcup \mathcal{M} = \{ 0,1,2,3\}$.
\end{description}

\feedbackright{TODO}

\feedbackfalse{TODO}
}
%END FRAGE

%BEGIN FRAGE
\frage{Sei $\mathcal{M} = \{ \{0,1,2\}, \{1,2\}, \{1,2,3\}, \{1,4\} \}$. Dann gelten:}

\antwort{
\typ{Multiple Choice, Kprim}
\level{leicht, mittel, schwer}
\begin{description}
\antwfalse $\bigcap \mathcal{M} = \emptyset$.
\antwright $\bigcap \mathcal{M} = \{ 1\}$.
\antwfalse $\bigcap \mathcal{M} = \{ 1,2\}$.
\antwfalse $\bigcap \mathcal{M} = \{ 0,1,2,3,4\}$.
\end{description}

\feedbackright{TODO}

\feedbackfalse{TODO}
}
%END FRAGE

\subsection*{Potenzmenge und Kardinalität}

%BEGIN FRAGE
\frage{Sei $M = \{ \emptyset,2, \{3\} \}$. Dann gelten:}

\antwort{
\typ{Multiple Choice, Kprim}
\level{leicht, mittel, schwer}
\begin{description}
\antwfalse $\mathcal{P}(M) = \{ \{\emptyset\}, \{ 2\}, \{ \{ 3 \} \}, \{\emptyset,2 \}, \{ \emptyset, \{ 3 \} \}, \{2,  \{ 3 \} \}, \{ \emptyset,2, \{ 3 \} \}    \}$.
\antwright $\mathcal{P}(M) = \{ \emptyset, \{\emptyset\}, \{ 2\}, \{ \{ 3 \} \}, \{\emptyset,2 \}, \{ \emptyset, \{ 3 \} \}, \{2,  \{ 3 \} \}, \{ \emptyset,2, \{ 3 \} \}    \}$.
\antwfalse $\mathcal{P}(M) = \{ \emptyset, \{\emptyset\}, \{ 2\}, \{ 3  \}, \{\emptyset,2 \}, \{ \emptyset,  3  \}, \{2,   3  \}, \{ \emptyset,2,  3  \}    \}$.
\antwfalse $\mathcal{P}(M) = \{  \{\emptyset\}, \{ 2\}, \{  3  \}, \{\emptyset,2 \}, \{ \emptyset,  3  \}, \{2,   3  \}, \{ \emptyset,2,  3  \}    \}$.

\end{description}

\feedbackright{TODO}

\feedbackfalse{TODO}
}
%END FRAGE


%BEGIN FRAGE
\frage{Es gelten:}

\antwort{
\typ{Multiple Choice, Kprim}
\level{leicht, mittel, schwer}
\begin{description}
\antwfalse $\mathcal{P}(\{\emptyset\}) = \{ \emptyset \}$.
\antwright $\mathcal{P}(\{\emptyset\}) = \{ \emptyset, \{\emptyset\} \}$.
\antwright $\mathcal{P}(\{\emptyset, \{\emptyset \} \}) = \{ \emptyset, \{\emptyset\}, \{\{\emptyset\}\}, \{\emptyset,\{\emptyset\}\} \}$.
\antwfalse $\mathcal{P}(\{\emptyset, \{\emptyset \} \}) = \{ \{\emptyset\}, \{\{\emptyset\}\}, \{\{\{\emptyset\}\}\}, \{\{\emptyset,\{\emptyset\}\}\} \}$.

\end{description}

\feedbackright{TODO}

\feedbackfalse{TODO}
}
%END FRAGE


%BEGIN FRAGE
\frage{Seinen $M = \{ 1,2,3,5\}$ und $N = \{0,1,4\}$. Dann gelten:}

\antwort{
\typ{Multiple Choice, Kprim}
\level{leicht, mittel, schwer}
\begin{description}
\antwfalse Dann gilt $\lvert M \cup N \rvert = 5$.
\antwright Dann gilt $\lvert M \cap N \rvert = 1$.
\antwright Dann gilt $\lvert \mathcal{P} (M) \rvert = 16$.
\antwright Dann gilt $\lvert \mathcal{P} (M \backslash N) \rvert = 8$.
\end{description}

\feedbackright{TODO}

\feedbackfalse{TODO}
}
%END FRAGE

%BEGIN FRAGE
\frage{Seinen $M$ und $N$ zwei disjunkte Mengen mit $\lvert M \rvert = 3$ und $\lvert N \rvert = 2$. Dann gelten:}

\antwort{
\typ{Multiple Choice, Kprim}
\level{leicht, mittel, schwer}
\begin{description}
\antwright Dann gilt $\lvert M \cup N \rvert = 6$.
\antwfalse Dann gilt $\lvert M \cap N \rvert = 1$.
\antwright Dann gilt $\lvert \mathcal{P} (M \cup N) \rvert = 64$.
\antwright Dann gilt $\lvert \mathcal{P} (M \backslash N) \rvert = 1$.
\end{description}

\feedbackright{TODO}

\feedbackfalse{TODO}
}
%END FRAGE


\subsection*{Relationen und Funktionen}

%BEGIN FRAGE
\frage{Welche der folgenden Relationen sind eindeutig?}

\antwort{
\typ{Multiple Choice, Kprim}
\level{leicht, mittel, schwer}
\begin{description}
\antwfalse $\{ (x,y) \in \mathbb{N} \times \mathbb{N} \ | \ x + y \leq 10 \}$.
\antwright $\{ (x,y) \in \mathbb{N} \times \mathbb{N} \ | \ x + y = 10 \}$.
\antwright $\{ (x,y) \in \mathbb{N} \times \mathbb{N} \ | \ x \cdot y = 120 \}$.
\antwright $\{ (x,y) \in \mathbb{Z} \times \mathbb{Z} \ | \ x \cdot y = 120 \}$.
\end{description}

\feedbackright{TODO}

\feedbackfalse{TODO}
}
%END FRAGE


%BEGIN FRAGE
\frage{Welche der folgenden Relationen sind total (bezüglich $\mathbb{N} \times \mathbb{N}$)?}

\antwort{
\typ{Multiple Choice, Kprim}
\level{leicht, mittel, schwer}
\begin{description}
\antwfalse $\{ (x,y) \in \mathbb{N} \times \mathbb{N} \ | \ x + y \leq 10 \}$.
\antwfalse $\{ (x,y) \in \mathbb{N} \times \mathbb{N} \ | \ x + y = 10 \}$.
\antwfalse $\{ (x,y) \in \mathbb{N} \times \mathbb{N} \ | \ x \cdot y = 120 \}$.
\antwright $\{ (x,y) \in \mathbb{N} \times \mathbb{N} \ | \ x \cdot 2 = y \}$.
\end{description}

\feedbackright{TODO}

\feedbackfalse{TODO}
}
%END FRAGE

%BEGIN FRAGE
\frage{Welche der folgenden Relationen sind Funktionen von $\mathbb{N}$ nach $\mathbb{N}$?}

\antwort{
\typ{Multiple Choice, Kprim}
\level{leicht, mittel, schwer}
\begin{description}
\antwfalse $\{ (x,y) \in \mathbb{N} \times \mathbb{N} \ | \ x < 10 \}$.
\antwfalse $\{ (x,y) \in \mathbb{N} \times \mathbb{N} \ | \ x < 10 \ \text{und} \ x = y \}$.
\antwright $\{ (x,y) \in \mathbb{N} \times \mathbb{N} \ | \ x \cdot 2 = y \}$.
\antwfalse $\{ (x,y) \in \mathbb{N} \times \mathbb{N} \ | \ 0,5 \cdot x  = y \}$.
\end{description}

\feedbackright{TODO}

\feedbackfalse{TODO}
}
%END FRAGE



%BACKUP

\subsection*{BACKUP}

%BEGIN FRAGE
\frage{text}

\antwort{
\typ{Single Choice, Multiple Choice, Kprim}
\level{leicht, mittel, schwer}
\begin{description}
\antwright richtig
\antwfalse falsch
\antwright richtig
\antwfalse falsch
\end{description}

\feedbackright{TODO}

\feedbackfalse{TODO}
}
%END FRAGE

